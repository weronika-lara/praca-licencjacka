Techniki omówione w poprzednim rozdziale doskonale sprawdzają się dla równań z zadanymi warunkami początkowymi. W tej części zajmiemy się jednak problemami opisanymi poprzez warunki brzegowe. Do tej klasy problemów stosowane są inne - lepiej dopasowane techniki. Część z nich rozszerza omówione wcześniej algorytmy. W tej części omówimy ogólną postać problemu z warunkami brzegowymi, jak i sposoby ich rozwiązywania za pomocą technik numerycznych. 


\begin{theorem}[O istnieniu rozwiązania \citep{kincaid1991numerical}]
Jeśli funkcja $\frac{df}{du}(t,u)$ jest ciągła, nieujemna i ograniczona w obszarze określonym nierównościami $0\leq t \leq 1$, $-\infty < u < +\infty$ to problem brzegowy postaci:
\begin{equation}\label{zagad_brzeg}
\left\{\begin{array}{ll}
u''(t)=f(t,u), & t \in (0,1) \\
u(0)=0, & \\
u(1)=0,
\end{array}\right.
\end{equation}
posiada jednoznaczne rozwiązanie w przedziale $[0,1]$.
\end{theorem}
Powyższe twierdzenie pokazuje, że istnienie rozwiązania problemu brzegowego może być rozstrzygnięte za pomocą odpowiednich warunków nałożonych na funkcję prawej strony równania. Nie wszystkie jednak problemy brzegowowe posiadają - najprostszy - zerowy warunek brzegowy oraz dziedzinę jaką jest przedział $(0,1)$. 

\begin{theorem}[{\citep{kincaid1991numerical}}]
Rozważmy następujące dwa problemy brzegowe:
\begin{equation}\label{pierwszy_problem_brzegowy}
\left\{\begin{array}{ll}
u''(t)=f(t,u(t)), & t \in (a,b) \\
u(a)=\alpha, & \\
u(b)=\beta,
\end{array}\right.
\end{equation}
\begin{equation}\label{drugi_problem_brzegowy}
\left\{\begin{array}{ll}
y''(t)=g(t,y(t)), &  t \in (0,1)\\
y(0)= \alpha, & \\
y(1)= \beta,
\end{array}\right.
\end{equation}
gdzie 
$$
g(p,q) = (b-a)^2f(a+(b-a)p,q).
$$
Wtedy jeśli $y$ jest rozwiązaniem \eqref{drugi_problem_brzegowy}, wówczas funkcja $u: t \mapsto y\left(\frac{t-a}{b-a}\right)$ jest rozwiązaniem problemu \eqref{pierwszy_problem_brzegowy}. Ponadto, jeśli $u$ jest rozwiązaniem \eqref{pierwszy_problem_brzegowy}, wówczas $y: t \mapsto u(a+(b-a)t)$ jest rozwiązaniem problemu \eqref{drugi_problem_brzegowy}.
\end{theorem}
\begin{proof}
Załóżmy, że $y$ jest rozwiązaniem \eqref{drugi_problem_brzegowy}. Wtedy funkcja $u$ postaci $u: t \mapsto y\left(\frac{t-a}{b-a}\right)$ spełnia warunki
\begin{align*}
u(a) &= y\left(\frac{a-a}{b-a}\right)= y(0) = \alpha, \\
u(b) &= y\left(\frac{b-a}{b-a}\right)= y(1) = \beta, \\
u'(t) &= y'\left(\frac{t-a}{b-a}\right)\frac{1}{b-a}, \\
u''(t) &= y''\left(\frac{t-a}{b-a}\right)\frac{1}{(b-a)^2} \\
&= g\left(\frac{t-a}{b-a},y\left(\frac{t-a}{b-a}\right)\right) \frac{1}{(b-a)^2}\\
&= (b-a)^2f\left(a+(b-a)\frac{t-a}{b-a},y\left(\frac{t-a}{b-a}\right)\right)\frac{1}{(b-a)^2}. \\
&= f(t,u(t)).
\end{align*}
Niech dalej $u$ będzie rozwiązaniem \eqref{pierwszy_problem_brzegowy}. Wtedy funkcja $y = u(a+(b-a)t)$ spełnia warunki
\begin{align*}
y(0) &= u(a+(b-a)\cdot 0) = u(a) = \alpha \\
y(1) &= u(a+(b-a)\cdot 1) = u(b) = \beta \\
y'(t) &= u'(a+(b-a)t)(b-a) \\
y''(t) &=  u''(a+(b-a)t)(b-a)^2 \\
&= f(a+(b-a)t, u(a+(b-a)t))(b-a)^2 \\
&= \frac{g(t,u(a+(b-a)t)}{(b-a)^2}(b-a)^2 \\
&= g(t,y(t))
\end{align*}
\end{proof}

Okazuje się, że istnieje również odpowiedniość w zakresie wartości na brzegu warunku - następującej postaci:
\begin{theorem}[{\citep{kincaid1991numerical}}]
Rozważmy problemy brzegowe postaci:
\begin{equation}\label{trzeci_problem_brzegowy}
\left\{\begin{array}{ll}
y''(t)=g(t,y(t)), t \in(0,1)& \\
y(0)=\alpha, & \\
y(1)=\beta,
\end{array}\right.
\end{equation}
\begin{equation}\label{czwarty_problem_brzegowy}
\left\{\begin{array}{ll}
z''(t)=h(t,z(t)), & t \in (0,1)\\
z(0)= 0, & \\
z(1)= 0,
\end{array}\right.
\end{equation}
gdzie 
$$
h(p,q)= g(p,q+\alpha + (\beta - \alpha)p).
$$
Jeśli $z$ jest rozwiązaniem \eqref{czwarty_problem_brzegowy}, to funkcja $y: t \mapsto z(t) + \alpha + (\beta - \alpha)t$ jest rozwiązaniem \eqref{trzeci_problem_brzegowy}. Co więcej jeżeli $y$ jest rozwiązaniem \eqref{trzeci_problem_brzegowy}, to $z: t \mapsto y(t) - [\alpha + (\beta - \alpha)t]$ jest rozwiązaniem \eqref{czwarty_problem_brzegowy}.
\end{theorem}
\begin{proof}
Niech $z$ będzie rozwiązaniem \eqref{czwarty_problem_brzegowy}. Wtedy funkcja $y: t \mapsto z(t) + \alpha + (\beta - \alpha)t$ spełnia warunki
\begin{align*}
y(0)&= z(0) + \alpha + (\beta - \alpha)0 = \alpha \\
y(1)&= z(1) + \alpha + (\beta - \alpha)1 = \beta \\
y''(t)&= z''(t) = h(t,z(t)) = g(t,z(t) + \alpha + (\beta - \alpha)t) \\
&= g(t,y(t)).
\end{align*}
Niech $y$ będzie rozwiązaniem \eqref{trzeci_problem_brzegowy}. Wtedy funkcja $z: t \mapsto y(t) - [\alpha + (\beta - \alpha)t]$ spełnia warunki 
\begin{align*}
z(0) &= y(0) - [\alpha + (\beta - \alpha)\cdot 0] = \alpha - \alpha = 0 \\
z(1) &= y(1) - [\alpha + (\beta - \alpha)\cdot 1] = \beta - \beta = 0 \\
z''(t) &= y''(t) = g(t,y(t)) = g(y, z(t) + [\alpha +(\beta - \alpha)t]) \\
&= h(t,z(t))
\end{align*} 
\end{proof}

Powyższe twierdzenia dają nam narzędzia do weryfikacji istnienia rozwiązania. Pokazują one, że dla dowolnego problemu brzegowego w typie \eqref{pierwszy_problem_brzegowy} istnieje zagadnienie w typie \eqref{czwarty_problem_brzegowy} dla którego istnienie rozwiązania wymusza istnienie rozwiązanie odpowiadającego problemu, zaś same rozwiązania są wzajemnie łatwe do wyznaczenia. 

\begin{theorem}[{\citep{kincaid1991numerical}}]
Jeżeli $f(t,s)$ jest funkcją ciągłą, gdzie $t \in [0,1]$ i $s \in \setR$ spełniającą warunek taki, że
$$
|f(t,s_1) - f(t,s_2)| \leq k|s_1 - s_2|, \quad k<8,
$$
to zagadnienie brzegowe postaci \eqref{zagad_brzeg} posiada jednoznaczne rozwiązanie ciągłe w $[0,1]$. 
\end{theorem}
Dokładniejszy szkic dowodu tego faktu może zostać odszukany w \citep{kincaid1991numerical}. Tu w pracy wspomnimy jedynie, że wykorzystuje on postać całkową równania, dla której to stosowane jest potem twierdzenie o punkcie stałym Banacha. Powyższe twierdzenie dodatkowo wykazuje, że rozwiązanie do tego problemu może posiadać rozwiązanie w klasie funkcji ciągłych. Warto odnotować, że przedstawione wcześniej transformacje opisujące zależności pomiędzy \eqref{pierwszy_problem_brzegowy} a \eqref{czwarty_problem_brzegowy}, nie zaburzają ciągłości - gdyż składają funkcje ciągłe z innymi funkcjami ciągłymi. 

\subsection{Metoda strzału}
W tej części przedstawimy jeden ze schematów rozwiązywania problemu brzegowego postaci:
\begin{equation}\label{metoda_strzalu}
\left\{\begin{array}{ll}
u''(t)=h(t,u(t),u'(t)), & \\
u(a)= \alpha, & \\
u(b)= \beta.
\end{array}\right.
\end{equation}
Schemat ten będziemy nazywali metodą strzału. 
\begin{algorithm}[Metoda Strzału]
Problem brzegowy postaci \eqref{metoda_strzalu} rozwiązujemy w następujący sposób:
\begin{enumerate}
\item Rozważamy zagadnienie początkowe postaci
\begin{equation} \label{eq-shooting-auxiliary}
\left\{\begin{array}{ll}
u''(t)=h(t,u(t),u'(t)), & \\
u(a)= \alpha, & \\
u'(a)= z.
\end{array}\right.
\end{equation}
\item Oznaczmy przez $\Phi$ funkcję przypisującą dla wybranego $z \in \setR$ funkcję będącą rozwiązaniem \eqref{eq-shooting-auxiliary}.
\item Rozwiązujemy równanie postaci 
$$
\Phi(z) = \beta.
$$ 
\end{enumerate}
Najtrudniejszym krokiem powyższego algorytmu jest końcowe rozwiązanie równania, z uwagi na bardzo skomplikowaną, a często niejawną postać funkcji $\Phi$. Do rozwiązywania problemów tą metodą wykorzystuje się kilka podejść - w szczególności
\begin{itemize}
\item Metodą siecznych,
\item Dekompozycję funkcji liniowej, czy
\item Metodę stycznych Newtona.
\end{itemize}
\end{algorithm}
W przypadku rozwiązań numerycznych zauważmy, że do wyznaczania wartości funkcji $\Phi$ używać możemy algorytmów przedstawionych w rozdziale 3.

Więcej o odmianach metody strzału odszukać można w pracy \citep{kincaid1991numerical}. Dalej przejdziemy do najbardziej nas interesującej metody różnic skończonych - wykorzystującej znane nam koncepcje z rozdziału \ref{chapter-Cauchy-problems}.
\subsection{Metoda różnic skończonych} \label{subsection-boundary-theory-finite-differences}
Wiemy, że (przy założeniu istnienia odpowiedniu wielu pochodnych rozwiązania) następujące warunki szacujące są prawdziwe:
\begin{equation}\label{roznice_skoncz_I}
u'(t) = \frac{1}{2h}[u(t+h) - u(t-h)] - \frac{1}{6}h^2u'''(\xi),
\end{equation}
oraz 
\begin{equation}\label{roznice_skoncz_II}
u''(t) = \frac{1}{h^2}[u(t+h) - 2u(t) + u(t+h)] - \frac{1}{12}h^2u^{(4)}(\tau).
\end{equation}
Jak przedtem zajmujemy się problemem brzegowym postaci \eqref{metoda_strzalu}. Podzielmy przedział $[a,b]$ punktami $ a=t_0<t_1<\cdots<t_N<t_{N+1}=b $. W praktyce przyjmuje się, że są one równoodległe $t_n = a + nh$ dla $n \in \set{0,\cdots,N+1}$. Niech $y_n$ oznacza przybliżoną wartość $u(t_n)$. Wersja dyskretna zadania oparta na wspomnianych wzorach jest następująca:
$$
\left\{\begin{array}{ll}
y_0 = \alpha, & \\
\frac{1}{h^2}(y_{n-1}-2y_n + y_{n+1}) = f\left(t_n,y_n,\frac{1}{2h}(y_{n+1} - y_{n-1})\right),& n \in \set{1,\cdots,N} \\
y_{n+1} = \beta.
\end{array}\right. 
$$
Powyższy układ $N$ równań z $N$ niewiadomymi $y_1,\cdots,y_N$ jest potencjalnie nieliniowy. Rozwiązywanie układu w tej postaci jest kłopotliwe. Zupełnie inaczej jest jednak w przypadku liniowym. Rozważmy zatem dalej szczególny przypadek gdzie:
$$
f(t,u,w) =f(t) +  g(t) u + m(t) w,
$$
Co pozwala na sprowadzenie naszego problemu do następującej postaci:
\begin{equation}
\left\{\begin{array}{ll}
u''(t)=f(t) + g(t) u(t) + m(t) u'(t), & t \in (a,b) \\
u(a)= \alpha, & \\
u(b)= \beta.
\end{array}\right.
\end{equation}
W przypadku tym układ równań do rozwiązania jest układem równań liniowych opisanych poniższymi zależnościami:
\begin{equation}\label{zagadnienie_liniowe}
\left\{\begin{array}{ll}
y_0 = \alpha, & \\
(1+ \frac{1}{2}h m(t_n))y_{n-1} - (2+ h^2 g(t_n))y_n + (1- \frac{1}{2}h m(t_n))y_{n+1} = h^2 f(t_n) ,& n \in \set{1,\cdots,N} \\
y_{N+1} = \beta,
\end{array}\right. 
\end{equation}
Wprowadżmy oznaczenia :
\begin{align*}
a_i:= 1 + \frac{1}{2}m(t_{i+1}),\\
d_i:= -(2+h^2 g(t_i)), \\
c_i:= 1 - \frac{1}{2}m(t_i), \\
b_i:= h^2 f(t_i).
\end{align*}
Wtedy powyższy układ równań możemy zapisać w postaci macierzowej:
\begin{equation}\label{macierz}
\left[ \begin{array}{ccccc}
d_1 & c_1 & 0 & 0 & 0 \\
a_1 & d_2 & c_2 & 0 & 0\\
\cdots & \cdots & \cdots &\cdots  &\cdots \\
0 & 0 & a_{N-2} & d_{N-1} & c_{N-1}\\
0 & 0 & 0 & a_{N-1} & d_N 
\end{array} \right] \cdot
\left[ \begin{array}{c}
y_1 \\
y_2 \\
\cdots \\
y_{N-1} \\
y_{N}  
\end{array} \right] =
\left[ \begin{array}{c}
b_1 - a_0\alpha\\
b_2 \\
\cdots \\
b_{n-1} \\
b_n -c_n\beta 
\end{array} \right] .
\end{equation}
Zauważmy też, że jeśli $g(t)>0$, a $h$ jest na tyle małe, że $ |h m(t)| \leq 2$, to ta macierz jest zdominowana przekątniowo. Istotnie, wtedy $ 1 \pm \frac{1}{2}h m(t) \geq 0 $ i
$$
|2 + h^2 g(t_n)| > |1 + \frac{1}{2}h m(t_n)| + |1 - \frac{1}{2}h^2 m(t_n)| = 2
$$

W tej dalszej części uzasadnimy, że dla $h \to 0$ rozwiązania zadania dyskretnego są zbieżne do rozwiązania problemu brzegowego postaci
\begin{equation}\label{zagad_brzeg_zbiez}
\left\{\begin{array}{cl}
u''(t) = f(t) + g(t) u(t) + m(t) u'(t), &  t \in (a,b)\\
u(a) = \alpha, & \\
u(b) = \beta.
\end{array}\right.
\end{equation}
Istnienie i jednoznaczność rozwiązania takiego zagadnienia wynika z następującego twierdzenia.
\begin{theorem}[{\citep{kincaid1991numerical}}]
Zagadnienie brzegowe postaci
\begin{equation}
\left\{\begin{array}{cl}
u''(t) = f(t,u(t),u'(t)), & \\
c_{11}u(a) + c_{12}u'(a) = c_{13}, & \\
c_{21}u(b) + c_{22}u'(b) = c_{23}
\end{array}\right.
\end{equation}
ma jednoznaczne rozwiązanie w przedziale $[a,b]$, jeśli
\begin{enumerate}
\item Funkcja $f$ i jej pierwsze pochodne cząstkowe $f_t$,$f_u$,$f_{u'}$ są ciągłe w obszarze $D:= [a,b] \times \setR \times \setR$.
\item $f_t >0, f_u \leq M$ i $|f_{u'}| \leq M$ w $D$.
\item $|c_{11}| + |c_{12}| > 0$, $|c_{21}| + |c_{22}| >0$ i $c_{11}c_{12}\leq 0 \leq c_{21}c_{22}$.
\end{enumerate}
\end{theorem}
\begin{proof}
Wróćmy do zagadnienia \eqref{zagad_brzeg_zbiez}. Niech $u$ będzie jego rozwiązaniem, a wektor $(y_1,\cdots,y_N)$ będzie rozwiązaniem układu \eqref{macierz}, oczywiście zależnym od $h$. Udowodnimy, że $|u(t_n) - y_n|$ dąży do $0$, gdy $h \to 0$. Z \eqref{roznice_skoncz_I} i \eqref{roznice_skoncz_II} wynika, że dla $ n \in \set{1\cdots,N}$
\begin{align*}
&\frac{1}{h^2} (u(t_{n-1}) - 2u(t_n) + u(t_{n+1})) - \frac{1}{12}h^2u^{(4)}(\tau_{n}) =\\
&f(t_n) + g(t_n)u(t_n) + m(t_n)\left[\frac{1}{2h}(u(t_{n+1}) - u(t_{n-1})) - \frac{1}{6}h^2u'''(\xi_n)\right].
\end{align*}
Natomiast rozwiązanie zadania dyskretnego spełnia równanie wynikające z poprzednich przez odrzucenie składników $u'''(t)$ i $u^{(4)}(t)$:
$$
\frac{1}{h^2}(y_{n-1} -2y_n + y_{n+1})= f(t_n)+g(t_n) y_n + m(t_n) \frac{1}{2h} (y_{n+1}+y_{n-1})
$$
Odejmujemy stronami odpowiednie równania z obu układów, przyjmując oznaczenie $e_n:=u(t_n) - y_n$
\begin{equation}
\frac{1}{h^2}(e_{n-1} - 2e_n + e_{n+1}) = g(t_n) e_n + m(t_n) \frac{1}{2h}(e_{n+1} - e_{n-1})+ h^2 v_n
\end{equation}
gdzie 
\begin{equation}\label{funkcja_g}
v_n:= \frac{1}{12} u^{(4)}(\tau_n) - m(t_n) \frac{1}{6} u'''(\xi_n).
\end{equation}
Po oczywistych przekształceniach otrzymujemy układ podobny do \eqref{zagadnienie_liniowe}:
$$
\begin{array}{cl}
\left(1+ \frac{1}{2}h m(t_n) \right)e_{n-1} - (2+ h^2 g(t_n) )e_n + (1- \frac{1}{2}h m(t_n) )e_{n+1} = h^4 v_n ,& n \in \set{1,\cdots,N}
\end{array}
$$
czyli zgodnie z przyjętymi oznaczeniami, układ postaci
$$
a_{n-1}e_{n-1} + d_ne_n + c_ne_{n+1} = h^4 v_n, \quad n \in \set{1,\cdots,N}.
$$
Stąd 
$$
|d_n||e_n| \leq h^4|v_n| + |c_n||e_{n+1}| + |a_{n-1}||e_{n-1}|.
$$
Niech $\lambda := {\norm{e}}_\infty$ i niech $n$ będzie takie, że $|e_n| = \lambda$. Dla tego $n$ zachodzi nierówność
$$
|d_n|\lambda \leq h^4 {\norm{v}}_\infty + |c_n|\lambda + |a_{n -1}|\lambda,
$$
czyli 
$$
\lambda(|d_n|-|c_n|-|a_{n-1}|) \leq h^4 {\norm{v}}_\infty.
$$
Stąd i z faktu,że $|d_n|-|c_n|-|a_{n-1}| = h^2 g(t_n)$ otrzymujemy:
$$
h^2 g(t_n) \lambda \leq h^4  {\norm{v}}_\infty, \quad {\norm{e}}_\infty \leq \frac{h^2 {\norm{v}}_\infty}{\inf g(t)}.
$$
Wobec \eqref{funkcja_g} jest
$$
\norm{v} \leq \frac{{\norm{u^{(4)}}}_\infty}{12}+\frac{{\norm{u'''}}_\infty}{6} \norm[\infty]{m}.
$$
Ta suma, jak i $\inf v(t)$ nie zależy od $h$, więc ${\norm{e}}_\infty = O(h^2)$ dla $h \to 0$ 
\end{proof}





