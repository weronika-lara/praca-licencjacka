Rozważmy następujące zagadnienie brzegowe. Niech $ x \in \Omega = (a,b) $
\begin{equation} \label{boundary_diff_equa}
\left\{ \begin{array}{ll}
-u_{xx}(x) + cu(x) =f(x), & x \in \Omega,\\
u(a) = \alpha, & \\
u(b) = \beta. & 
\end{array} \right.
\end{equation}
gdzie stała $c$ jest nieujemna, odcinek $ [a,b]$ jest ustalony i znane są wartości $\alpha$ oraz $\beta$. Wykorzystajmy wzór na różnicę skończoną w przód i różnicę skończoną w tył z definicji \ref{operator_wporzd} oraz \ref{operator_wtyl} . Dla $h>0$ :
\begin{equation} \label{finite_diff}
\begin{array}{c}
\delta_h u(x) = \frac{u(x+h) - u(x)}{h},   \\
\overline{\delta_h} u(x) = \frac{u(x)-u(x-h)}{h} . 
\end{array}
\end{equation}
Jeżeli $h$ będzie miało ustaloną wartość to będziemy opuszczali indeks dolny. Przyjmijmy następujące przybliżenie drugiej pochodnej:
$$
-u_{xx}(x) \approx -\delta \overline{\delta} u(x).
$$
Podstawmy \eqref{finite_diff} do \eqref{boundary_diff_equa}. Wówczas dla ustalonego $h>0$ mamy: 
\begin{equation}
\begin{array}{rl}
-\delta (\frac{u(x)-u(x-h}{h}) + cu(x) &= f(x),  \\
- \frac{1}{h} (\delta u(x) - \delta u(x-h) ) + cu(x) &= f(x), \\
-\frac{1}{h} (\frac{u(x+h)-u(x)}{h} - \frac{u(x)-u(x-h)}{h}) + cu(x) &= f(x), \\
-\frac{u(x+h)}{h^2} + \frac{u(x)}{h^2} + \frac{u(x)}{h^2} - \frac{u(x-h)}{h^2} +cu(x)& = f(x), \\
-\frac{u(x+h)}{h^2} + 2\frac{u(x)}{h^2} - \frac{u(x-h)}{h^2} +cu(x) &= f(x) .
\end{array}
\end{equation}
Mnożąc obie strony przez $-h^2$ otrzymujemy:
\begin{equation}
u(x+h) - 2u(x) + u(x-h) - cu(x)h^2 = -f(x)h^2,
\end{equation}
wprowadźmy siatkę, czyli zbiór dyskretny dla $ h= \frac{b-a}{N}$ , $k \in \set{1,\cdots,N-1}$. Mamy: 
\begin{equation} \label{boundary_model}
u(x_{k+1}) = 2u(x_k) - u(x_{k-1}) + cu(x_k)h^2 + f(x_k)h^2.
\end{equation}
W zagadnieniach brzegowych nie jest znana wartość $u(x_1)$. Zauważmy, że gdy rozważymy jej wartość $u(x_1) = s \in \setR$, to stosując nasz schemat możemy obliczyć wszystkie kolejne iteracje. Zatem nasze zadanie wymagać będzie rozwiązania problemu odnalezienia takiego $s$, że $u(x_1) = s$ i $u(x_k) = \beta$. Przyjmując oznaczenie $ x_k = x_0 + kh$, dla $ k \in \set{1, \cdots,N-1} $ otrzymujemy układ równań liniowych postaci:
\begin{equation}
\left\{ \begin{array}{ll}
u(x_0) = \alpha ,& \\
u(x_2) = 2u(x_1) - u(x_{0}) + cu(x_1)h^2 + f(x_1)h^2 ,& \\ 
\vdots  & \\
u(x_{N}) = 2u(x_{N-1}) - u(x_{N-2}) + cu(x_{N-1})h^2 + f(x_{N-1})h^2 , & \\
u(x_N) = \beta. &
\end{array} \right.
\end{equation}
Otrzymaliśmy układ $N+1$ równań liniowych z $N+1$ niewiadomymi. Problem w obecnej postaci jesteśmy wstanie rozwiązać, o ile otrzymana macierz jest nieosobliwa. Zapiszmy ten układ za pomocą postaci macierzowej :
$$ 
\left[ \begin{array}{ccccc}
1 & 0 & 0 & \cdots &0 \\
1 & -2-ch^2 & 1 &\cdots &0 \\
0 & 1 & -2-ch^2 &\cdots &0 \\
\vdots & \vdots & \vdots &  &\vdots \\
0& 0 & \cdots& 1 & -2-ch^2 \\
0 & 0 & 0&\cdots & 1 
\end{array} \right] \cdot
\left[ \begin{array}{c}
u_0 \\
u_1 \\
u_2 \\
\vdots \\
u_{N-1} \\
u_{N}  
\end{array} \right] =
\left[ \begin{array}{c}
\alpha \\
-f(x_1)h^2 \\
-f(x_2)h^2\\
\vdots \\
-f(x_{N}) \\
\beta   
\end{array} \right] .
$$
Warto zauważyć, że możemy usunąć pierwszy i ostatni wiersz otrzymując  wtedy :
$$
\left[ \begin{array}{ccccc}
-2-ch^2 & 1 & 0 & \cdots & 0 \\
1 & -2-ch^2 & 1& \cdots & 0 \\
0 & 1 & -2-ch^2& \cdots & 1 \\
\vdots &  & \ddots & \ddots & \vdots \\ 
0 & 0 & \cdots & 1 & -2-ch^2 \\
\end{array} \right] \cdot
\left[ \begin{array}{c}
u_1 \\
u_2 \\
u_3 \\
\vdots \\
u_{N-1} 
\end{array} \right] =
\left[ \begin{array}{c}
-f(x_1)-\alpha \\
-f(x_2)h^2 \\
-f(x_3)h^2\\
\vdots \\
-f(x_{N-1})h^2 -\beta    
\end{array} \right] . 
$$
Rozwiązując powyższy układ otrzymamy taką wartość $u(x_1)$, że $u(x_N) = \beta$. Powyższy algorytm nazywać będziemy schematem brzegowym dla równań liniowych rzędu drugiego. 