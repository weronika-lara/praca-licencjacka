\documentclass[12pt,a4paper]{report}
\usepackage[utf8]{inputenc}
\usepackage{amsmath}
\usepackage{amsfonts}
\usepackage{amssymb}
\usepackage{polski}
\usepackage{url}
\usepackage{hyperref}
\usepackage[left=2cm,right=2cm,top=2cm,bottom=2cm]{geometry}
\usepackage{natbib} %bibtex!!
\newtheorem{definition}{Definicja}
\author{Weronika Lara}
\title{Numeryczne rozwiązywanie równań różniczkowych drugiego rzędu z warunkami brzegowymi}
\begin{document}

\maketitle

\chapter{Wstęp}

Cytując \citep[Rozdział 3, sekcja 2]{palczewski2004rownania}

\chapter{Preliminaria}
\begin{definition}
Niech $F : R \times R^{n+1} \to R$ . Postacią ogólną równania różniczkowego nazywamy równianie: 
\begin{equation*}
F(t, x(t), x'(t),..., x^{(n)}(t)) = 0
\end{equation*} 
Rzędem równania nazywamy wtedy liczbę naturalą $n$. 
\end{definition}
\begin{definition}
Rozwiązaniem równania różniczkowego nazywamy funkcję $y:(a, b)\times R$ różniczkowalna tyle razy, ile wynosi rząd równania i taka, że dla dowolnego $t (a, b)$
\begin{equation*}
F(t, y(t), y'(t),..., y^{(n)}(t)) = 0
\end{equation*} 
\end{definition}
\begin{definition}  







\chapter{Metoda Eulera}
Najprostszą metodą numeryczną rozwiązywania równań różniczkowych jest metoda Eulera. Przybliżamy pochodną poprzez iloraz różnicowy dla pewnego
parametru $h > 0$ 
\begin{equation*}
\frac{x(t+h) - x(t)}{h} \approx \frac{dx}{dt}
\end{equation*}
\newline
i otrzymujemy otwarty schemat Eulera:
\begin{equation*}
\frac{x_h(t+h)-x_h(t)}{h} = f(t,x(t))
\end{equation*}
\newline 
czy inaczej 
\begin{equation*}
x_h(t+h) = x_h(t) + hf(t,x(t))
\end{equation*}
\newline
znając rozwiązanie w punkcie $t_0 x_h(t0) = x_0$ możemy wyznaczyć przybliżone rozwiązanie
$xh$ w kolejnych punktach $t_n = t_0 + nh$ z powyższego wzoru. Ale można też przybliżyć
pochodną biorąc parametr $−h$ w tył:
\begin{equation*}
\frac{x_h(t)-x_h(t-h)}{h} \approx \frac{dx}{dt}
\end{equation*}
\newline
i wtedy zastępując pochodną przez taki iloraz otrzymujemy zamknięty schemat Eulera:
\begin{equation*}
\frac{x_h(t)-x_h(t-h)}{h} = f(t,x_h(t))
\end{equation*}
\newline
czy inaczej 
\begin{equation*}
x_h(t+h) = x_h(t) + hf(t,x_h(t+h))
\end{equation*}
\newline
Warto zauważyć, że jeśli znamy rozwiązanie w punkcie $t_0$, tzn.$ x_0$) = $x_0$, to aby wyznaczyć kolejne przybliżenia rozwiązania w punktach $t = t_n = t_0+nh$ należy rozwiązać równania postaci:
\begin{equation*}
g(y):= y -hf(t, y) - x_h(t) = 0$
\end{equation*}
\newline
względem $y$, co sprawia, że zamknięty schemat Eulera może wydać się mało praktyczny w porównaniu z otwartym schematem Eulera. Dla niektórych równań jest to pozorne. Zauważmy tylko, że im $h$ mniejsze, tym potencjalnie powyższe równanie jest łatwiejsze do rozwiązania. 
\chapter{Schemat  Rungego-Kutty}
Podstawową klasą schematów jednokrokowych są tzw. schematy Rungego-Kutty. 
Załóżmy, że znamy $x_n$, i chcemy wyliczyć wartość $x_{n+1}$ ze wzoru uwzględniającego wartość
pola wektorowego nie tylko w $x_n$, ale również w dodatkowym punkcie 
$x^{\sim}$. Wtedy
\begin{equation*}
x_{n+1} = F(h,t_n,x_n,x^{\sim})
\end{equation*}
\newline
Biorąc schemat otwarty Eulera z krokiem $h^{\sim}$ otrzymujemy punkt
\begin{equation*}
x^{\sim} = xn + h f^{\sim}(t,x(t))
\end{equation*}
\newline
który, jak wiemy, przybliża $x(t+h^{\sim})$, ale niedokładnie. Możemy policzyć wartość pola wektorowego $f$ w tym punkcie i następnie, wykorzystując wartość $y_n = f(t_n, x_n)$ i $y^{\sim} = f(t_{n+h},x^{\sim})$, znaleźć lepsze przybliżenie $x(t_{n+1})$ - czyli np. za przybliżenie pola wektorowego wziąć ważoną średnią obu wartości $by_n+cy^{\sim}$ dla pewnych ustalonych wag $b$,$c$. Możliwości
jest wiele. Pojawia się pytanie: jak oceniać różne konstrukcje $F$? Można tak dobierać $F$, aby rząd schematu był możliwie duży. 
\newline
Załóżmy, że $h^{\sim}=ah$. Wtedy szukamy schematu postaci:
\begin{equation*}
x_{n+1} = xn + bhf_n + chf(t_n + ah, x_n + ahf(t_n,x_n))
\end{equation*}
\newline
tak, aby schemat miał maksymalny rząd.
\newline
Rozwijamy rozwiązanie $x$ w szereg Taylora:







\chapter{Cześć praktyczna}
% prezentacja teorii w praktyce

\chapter{Podsumowanie}

\bibliographystyle{plain}
\bibliography{bibliografia}

\end{document}